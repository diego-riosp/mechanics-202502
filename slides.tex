\begin{frame}
    \begin{center}
        \Huge ¡Bienvenidos!
    \end{center}
\end{frame}

\begin{frame}{Matrícula inicial}
    \begin{figure}
        \centering
        \includegraphics[width=0.8\linewidth]{figures/matricula-inicial.png}
    \end{figure}
\end{frame}

\begin{frame}{Generalidades}
    \begin{itemize}
        \item Horario de clases: martes y jueves 20:00 $-$ 22:00
        \item Clases en Zoom
        
        \begin{itemize}
            \item Enlace: {\color{blue}\url{udearroba.zoom.us/j/93686452220}}
            \item Meeting ID: 936 8645 2220
        \end{itemize} 
        \item Las sesiones se dividirán en:
        \begin{itemize}
            \item Teoría: aproximadamente una hora
            \item Práctica: aproximadamente 40 minutos
        \end{itemize}
        \item Horario de asesorías: martes 14:00 $-$ 15:00
        \begin{itemize}
            \item Virtual: {\color{blue}\url{meet.google.com/dgr-qdvk-qpr}}
            \item Presencial: sede Medellín Ciudad Universitaria, aula 6-120.
        \end{itemize}
        
    \end{itemize}
\end{frame}

\begin{frame}{Generalidades}
    \begin{itemize}
        \item Textos guía:
        \begin{itemize}
            \item Principal: Sears \& Zemansky (2013), \textit{Física Universitaria}. 13ava Ed., Vol. 1.
            \item Secundario: Serway \& Jewett (2008), \textit{Física para ciencias e ingeniería}. 7ma Ed., Vol. 1.
        \end{itemize}
        \item Esta presentación se actualizará clase a clase en los activos del repositorio {\color{blue}\url{github.com/diego-riosp/mechanics-202502}}.
    \end{itemize}
\end{frame}

\begin{frame}{Generalidades}
    Se propone que el tiempo de trabajo a la semana se divida de la siguiente forma:
\begin{itemize}
    \item 4 horas de sesiones de clase con el profesor
    \item 1 hora de lectura previa a la sesión de clase de las correspondientes sesiones.
    \item 1 hora de lectura posterior a la sesión de clase examinando los ejemplos del texto guía y enfatizando en los conceptos no comprendidos.
    \item 2 horas para el desarrollo de las preguntas y ejercicios planteados.
    \item 1 hora para leer materiales de apoyo, desarrollar las guías de estudio o realizar la autoevaluación en Ingeni@.
\end{itemize}
\end{frame}

\begin{frame}{Evaluación}
\begin{table}[]
    \centering
    \begin{tabular}{|c|c|c|c|c|}
    \hline
     \textbf{Examen} &  \textbf{Porcentaje (\%)}  &  \textbf{Fecha}& \textbf{Modalidad}\\\hline
     Quiz 1 & 5  &  04/09& Virtual\\\hline
     Parcial 1 & 20  &  09/09& Virtual\\\hline
     Quiz 2 & 5  & 09/10& Virtual \\\hline
     Parcial 2 & 20  & 11/10& Presencial\\\hline
     Quiz 3 & 5  & 11/11& Virtual\\\hline
     Parcial 3 (Oral) & 20 & 13/11& Virtual\\\hline
     Quiz 4 & 5  & 27/11& Virtual\\\hline
     Parcial 4 & 20  & 29/11& Presencial\\\hline
\end{tabular}
\end{table}

Los exámenes parciales 1 y 3 se realizarán en el horario de clase y en la fecha según la programación.

Los exámenes parciales 2 y 4 se realizarán en cada una de las sedes y seccionales según la programación, en el horario 14:00 $-$ 16:00.

\end{frame}

\begin{frame}
    \begin{center}
        \Huge ¿Preguntas?
    \end{center}
\end{frame}

\begin{frame}

    \begin{figure}
        \centering
        \includegraphics[width=0.5\linewidth]{figures/clasical-physics.jpg}
    \end{figure}
    
\begin{center}
    \LARGE Hagamos Física Clásica
\end{center}
    
\end{frame}

\begin{frame}
\begin{center}
    \Huge \textbf{Capítulo 1}
    
    \LARGE Unidades, cantidades físicas y vectores

    \textit{Comencemos desde el principio}
\end{center}
\end{frame}

\begin{frame}{Ideas}
    \begin{enumerate}
        \item La física es una ciencia experimental
        \item Un número empleado para describir cuantitativamente un fenómeno físico es una cantidad física
        \item Al medir una cantidad, siempre la \textit{comparamos} con un estándar de referencia. Si decimos que un Ferrari tiene una longitud de 4.53 m, queremos decir que es 4.53 veces más largo que una vara de cierto tamaño (1 m).
    \end{enumerate}
\end{frame}

\begin{frame}{¿Qué se compara al medir?}
    \begin{figure}
        \centering
        \includegraphics[width=0.8\linewidth]{figures/que-se-mide.png}
    \end{figure}

    Por ejemplo, para medir distancia, comparemos con la luz. Para medir masas, comparemos con objetos que nunca más volveremos a tocar. Para medir tiempo, comparemos con átomos.    
\end{frame}

\begin{frame}{¿Qué podemos medir?}

\textbf{Magnitud física}

    Una magnitud física es toda propiedad de un fenómeno, cuerpo o sustancia que se puede medir y expresar con un número y una unidad.
    
    \vspace{1em}

    \begin{columns}
        \column{0.45\textwidth}
        \textbf{Fundamentales}
        
        Se definen por sí mismas
        \begin{itemize}
            \item Tiempo
            \item Longitud
            \item Masa
            \item Temperatura
            \item Cantidad de sustancia
            \item Intensidad luminosa
            \item Carga eléctrica
        \end{itemize}
        $$\vdots$$
        
        \column{0.45\textwidth}
        \textbf{Derivadas}
        
        Se obtienen a partir de las fundamentales
        \begin{itemize}
            \item Velocidad
            \item Aceleración
            \item Fuerza
            \item Potencia
            \item Energía
            \item Corriente eléctrica
        \end{itemize}
        $$\vdots$$
    \end{columns}
\end{frame}

\begin{frame}{¿Qué referencias usamos para medir?}
\textbf{Unidades de medida}

Las unidades de medida son estándares o referencias que se usan para expresar el valor de una magnitud física.

\vspace{1em}

\textit{Ejemplos:} metros, pulgadas, yardas, segundos, eones, años, gramos, coulombs, newtons, amperios, voltios, celcius, parsec, años luz, candelas, etc.

\end{frame}

\begin{frame}{¿Qué estándares usamos para medir?}
    \textbf{Sistemas de medida}

Un sistema de medida es un conjunto organizado de unidades de medida y reglas que se usan para medir y expresar magnitudes físicas de forma uniforme.

Es como un “idioma común” para las mediciones: define qué unidades usar y cómo relacionarlas.

    \begin{table}[h]
\centering
\caption{Unidades base del Sistema Internacional (SI)}
\begin{tabular}{|l|l|l|}
\hline
\textbf{Magnitud} & \textbf{Unidad} & \textbf{Símbolo} \\ \hline
Longitud & metro & m \\ \hline
Masa & kilogramo & kg \\ \hline
Tiempo & segundo & s \\ \hline
Temperatura & kelvin & K \\ \hline
Corriente eléctrica & amperio & A \\ \hline
Cantidad de sustancia & mol & mol \\ \hline
Intensidad luminosa & candela & cd \\ \hline
\end{tabular}
\end{table}
\end{frame}

\begin{frame}
    \begin{table}[h]
\centering
\caption{Unidades comunes en el sistema inglés}
\begin{tabular}{|l|l|l|}
\hline
\textbf{Magnitud} & \textbf{Unidad} & \textbf{Símbolo} \\ \hline
Longitud & pie & ft \\ \hline
Longitud & pulgada & in \\ \hline
Masa & libra & lb \\ \hline
Tiempo & segundo & s \\ \hline
Temperatura & grado Fahrenheit & $^\circ$F \\ \hline
Fuerza & libra-fuerza & lbf \\ \hline
Velocidad & milla por hora & mph \\ \hline
\end{tabular}
\end{table}

\end{frame}

\begin{frame}{Equivalencias entre sistemas}
\footnotesize
    \begin{table}[h]
\centering
\caption{Equivalencias entre el Sistema Internacional y el Sistema Inglés}
\begin{tabular}{|l|l|l|}
\hline
\textbf{Magnitud} & \textbf{Sistema Internacional (SI)} & \textbf{Sistema Inglés} \\ \hline
Longitud & 1 metro (m) = 3.2808 pies (ft) & 1 pie (ft) = 0.3048 m \\ \hline
Longitud & 1 kilómetro (km) = 0.6214 millas (mi) & 1 milla (mi) = 1.6093 km \\ \hline
Masa & 1 kilogramo (kg) = 2.2046 libras (lb) & 1 libra (lb) = 0.4536 kg \\ \hline
Fuerza & 1 newton (N) = 0.2248 libras-fuerza (lbf) & 1 lbf = 4.4482 N \\ \hline
Velocidad & 1 m/s = 2.2369 millas/hora (mph) & 1 mph = 0.4470 m/s \\ \hline
Temperatura & $^\circ$C = ($^\circ$F - 32) × 5/9 & $^\circ$F = ($^\circ$C × 9/5) + 32 \\ \hline
\end{tabular}
\end{table}

\end{frame}

\begin{frame}{Ejemplos}

\textbf{Ejemplo 1: Inglés $\rightarrow$ SI}  

Convertir \SI{12}{\foot} a metros:

\[
\SI{12}{\foot} \times \frac{\SI{0.3048}{\metre}}{\SI{1}{\foot}} 
= \SI{3.6576}{\metre}
\]

\textbf{Ejemplo 2: SI $\rightarrow$ Inglés}  

Convertir \SI{5}{\metre} a pies:

\[
\SI{5}{\metre} \times \frac{\SI{3.2808}{\foot}}{\SI{1}{\metre}} 
= \SI{16.404}{\foot}
\]
\end{frame}

\begin{frame}{¿Cómo evitamos números muy largos?}

\textbf{Prefijos}

    Los prefijos son partículas que se colocan delante del nombre o símbolo de una unidad para indicar que la magnitud medida es un múltiplo o un submúltiplo de esa unidad.

En otras palabras, los prefijos nos ahorran escribir muchos ceros, tanto a la izquierda como a la derecha de la coma decimal.
\end{frame}

\begin{frame}

\footnotesize

    \begin{table}[h]
\centering
\begin{tabular}{|l|l|l|}
\hline
\textbf{Prefijo} & \textbf{Símbolo} & \textbf{Factor de multiplicación} \\ \hline
yotta  & Y  & $10^{24}$  \\ \hline
zetta  & Z  & $10^{21}$  \\ \hline
exa    & E  & $10^{18}$  \\ \hline
peta   & P  & $10^{15}$  \\ \hline
tera   & T  & $10^{12}$  \\ \hline
giga   & G  & $10^{9}$   \\ \hline
mega   & M  & $10^{6}$   \\ \hline
kilo   & k  & $10^{3}$   \\ \hline
hecto  & h  & $10^{2}$   \\ \hline
deca   & da & $10^{1}$   \\ \hline
—      & —  & $10^{0}$   \\ \hline
deci   & d  & $10^{-1}$  \\ \hline
centi  & c  & $10^{-2}$  \\ \hline
milli  & m  & $10^{-3}$  \\ \hline
micro  & $\mu$ & $10^{-6}$  \\ \hline
nano   & n  & $10^{-9}$  \\ \hline
pico   & p  & $10^{-12}$ \\ \hline
femto  & f  & $10^{-15}$ \\ \hline
atto   & a  & $10^{-18}$ \\ \hline
zepto  & z  & $10^{-21}$ \\ \hline
yocto  & y  & $10^{-24}$ \\ \hline
\end{tabular}
\end{table}
\end{frame}

\begin{frame}{Ejemplos}
    \begin{itemize}
    \item \SI{1}{\kilo\metre} $\;=\;$ 1000 metros $\;=\;$ $10^{3} \ \si{\metre}$
    \item \SI{1}{\milli\metre} $\;=\;$ 0.001 metros $\;=\;$ $10^{-3} \ \si{\metre}$
    \item \SI{1}{\micro\second} $\;=\;$ 0.000001 segundos $\;=\;$ $10^{-6} \ \si{\second}$
\end{itemize}
\end{frame}

\begin{frame}
    \begin{center}
        \Huge ¿Preguntas?
    \end{center}
\end{frame}

\begin{frame}
    Para resolver los ejercicios que a continuación se presentan, implemente las siguientes equivalencias.

    \begin{align}
        \num{1}\,\unit{in} &= \num{2.54}\,\unit{cm}\\
        \num{1}\,\unit{ft} &= \num{30.48}\,\unit{cm}\\
        \num{1}\,\unit{yd} &= \num{0.91}\,\unit{m}\\
        \num{1}\,\unit{mi} &= \num{1.609}\,\unit{km}\\
        \num{1}\,\unit{lb} &= \num{453.59}\,\unit{g}\\
        \num{1}\,\unit{fl oz} &= \num{29.57}\,\unit{ml}\\
        \num{1}\,\unit{ton} &= \num{907.19}\,\unit{kg}\\
        \num{1}\,\unit{oz} &= \num{28.35}\,\unit{kg}\\
        \num{1}\,\unit{gal} &= \num{3.79}\,\unit{l}
    \end{align}
\end{frame}

\begin{frame}{Ejercicios propuestos}
    \begin{multicols}{3}
    \begin{enumerate}
    \item $\num{15} \,\unit{in}\rightarrow\unit{cm}$
    \item $\num{3.5} \,\unit{ft}\rightarrow\unit{cm}$
    \item $\num{2.1} \,\unit{yd}\rightarrow\unit{in}$
    \item $\num{12} \,\unit{mi}\rightarrow\unit{cm}$
    \item $\num{450} \,\unit{lb}\rightarrow\unit{ton}$
    \item $\num{8} \,\unit{lb}\rightarrow\unit{g}$
    \item $\num{56} \,\unit{oz}\rightarrow\unit{g}$
    \item $\num{10} \,\unit{gal}\rightarrow\unit{l}$
    \item $\num{5} \,\unit{m}\rightarrow\unit{in}$
    \item $\num{1.8} \,\unit{cm}\rightarrow\unit{ft}$
    \item $\num{3.2} \,\unit{m}\rightarrow\unit{yd}$
    \item $\num{25} \,\unit{km}\rightarrow\unit{mi}$
    \item $\num{0.75} \,\unit{ton}\rightarrow\unit{kg}$
    \item $\num{500} \,\unit{g}\rightarrow\unit{oz}$
    \item $\num{100} \,\unit{g}\rightarrow\unit{lb}$
    \item $\num{20} \,\unit{l}\rightarrow\unit{gal}$
    \item $\num{48} \,\unit{in}\rightarrow\unit{cm}$
    \item $\num{7} \,\unit{ft}\rightarrow\unit{cm}$
    \item $\num{6} \,\unit{yd}\rightarrow\unit{m}$
    \item $\num{35} \,\unit{mi}\rightarrow\unit{km}$
    \item $\num{2} \,\unit{ton}\rightarrow\unit{kg}$
    \item $\num{2.5} \,\unit{lb}\rightarrow\unit{g}$
    \item $\num{5} \,\unit{oz}\rightarrow\unit{g}$
    \item $\num{15} \,\unit{gal}\rightarrow\unit{l}$
    \item $\num{2.4} \,\unit{cm}\rightarrow\unit{ft}$
    \item $\num{8} \,\unit{m}\rightarrow\unit{yd}$
    \item $\num{36} \,\unit{in}\rightarrow\unit{ft}$
    \item $\num{5.5} \,\unit{ft}\rightarrow\unit{yd}$
    \item $\num{12} \,\unit{yd}\rightarrow\unit{mi}$
    \item $\num{5280} \,\unit{ft}\rightarrow\unit{mi}$
    \item $\num{96} \,\unit{oz}\rightarrow\unit{lb}$
    \item $\num{120} \,\unit{lb}\rightarrow\unit{ton}$
    \item $\num{0.75} \,\unit{ton}\rightarrow\unit{lb}$
    \item $\num{256} \,\unit{oz}\rightarrow\unit{ton}$
    \item $\num{192} \,\unit{fl\,oz}\rightarrow\unit{gal}$
    \item $\num{144} \,\unit{in^2}\rightarrow\unit{ft^2}$
    \item $\num{9} \,\unit{ft^2}\rightarrow\unit{yd^2}$
    \item $\num{27} \,\unit{ft^3}\rightarrow\unit{yd^3}$
    \item $\num{1728} \,\unit{in^3}\rightarrow\unit{ft^3}$
    \item $\num{1760} \,\unit{yd}\rightarrow\unit{mi}$
    \item $\num{2.5} \,\unit{mi}\rightarrow\unit{yd}$
    \end{enumerate}
\end{multicols}
\end{frame}

\begin{frame}
\setlength{\columnsep}{0.5cm} %
\footnotesize
    \begin{multicols}{2}
    \begin{enumerate}
    \item $\num{6728.73004} \,\unit{km}\rightarrow\unit{dm}$
    \item $\num{7526859842.59} \,\unit{mg}\rightarrow\unit{hg}$
    \item $\num{0.000000598} \,\unit{\mu s}\rightarrow\unit{fs}$
    \item $\num{59863.254701} \,\unit{Glb}\rightarrow\unit{Mlb}$
    \item $\num{2.0000256} \,\unit{Ymol}\rightarrow\unit{Tmol}$
    \item $\num{0.0000000000000001} \,\unit{Tm}\rightarrow\unit{nm}$
    \item $\num{100000000000000000} \,\unit{cm}\rightarrow\unit{Zm}$
    \item $\num{0.000236589725} \,\unit{hs}\rightarrow\unit{ps}$
    \item $\num{0.0002555} \,\unit{cm}\rightarrow\unit{dam}$
    \item $\num{560029698.2256301} \,\unit{kb}\rightarrow\unit{Gb}$
    \item $\num{1} \,\unit{Y^\circ C}\rightarrow\unit{y^\circ C}$
    \item $\num{0.00000003265871} \,\unit{\mu K}\rightarrow\unit{kK}$
    \item $\num{5897.003} \,\unit{mL}\rightarrow\unit{L}$
    \item $\num{0.00001254} \,\unit{Gm}\rightarrow\unit{km}$
    \item $\num{9547863.25} \,\unit{cm^3}\rightarrow\unit{m^3}$
    \item $\num{0.000000025} \,\unit{Pb}\rightarrow\unit{Tb}$
    \item $\num{325.698} \,\unit{g}\rightarrow\unit{mg}$
    \item $\num{0.00002536} \,\unit{ds}\rightarrow\unit{ms}$
    \item $\num{95872365.4} \,\unit{dm^2}\rightarrow\unit{km^2}$
    \item $\num{0.0000000000147} \,\unit{Ts}\rightarrow\unit{Gs}$
    \item $\num{1.254} \,\unit{pm}\rightarrow\unit{fm}$
    \item $\num{325.698} \,\unit{hl}\rightarrow\unit{ml}$
    \item $\num{9587.002} \,\unit{km}\rightarrow\unit{m}$
    \item $\num{0.000000000000365} \,\unit{Ms}\rightarrow\unit{ns}$
    \item $\num{123.589} \,\unit{fl}\rightarrow\unit{pl}$
    \item $\num{0.53248} \,\unit{kb^2}\rightarrow\unit{db^2}$
\end{enumerate}
\end{multicols}
\end{frame}

\begin{frame}{Incertidumbre}
    Toda medición tiene incertidumbre, la cual depende del instrumento y la técnica utilizada.

\vspace{1em}
    
\textbf{Un ejemplo:} medir con una regla común da un espesor de 3 mm con una precisión solo al milímetro más cercano, mientras que un micrómetro da 2.91 mm con precisión al 0.01 mm.

\vspace{1em}

\textit{La medición con menor incertidumbre es más exacta}.
\end{frame}

\begin{frame}{Cifras significativas}
    Las \textbf{cifras significativas} son los dígitos de una medición que aportan 
información real sobre su valor, es decir, aquellos que se conocen con certeza 
más el primer dígito incierto.

\vspace{1em}

Sirven para \textbf{indicar la precisión} de una medición y dependen tanto del 
instrumento usado como de la forma de registrar el dato.
\end{frame}

\begin{frame}
    \textbf{Reglas básicas para identificarlas}
\begin{enumerate}
    \item \textbf{Todos los dígitos distintos de cero} son significativos.  
    Ej.: $345$ $\rightarrow$ 3 cifras significativas.
    
    \item \textbf{Los ceros entre dígitos distintos de cero} son significativos.  
    Ej.: $2007$ $\rightarrow$ 4 cifras significativas.
    
    \item \textbf{Los ceros a la izquierda} no son significativos 
    (solo indican posición decimal).  
    Ej.: $0.0045$ $\rightarrow$ 2 cifras significativas.
    
    \item \textbf{Los ceros a la derecha} son significativos si hay punto decimal.  
    Ej.: $45.00$ $\rightarrow$ 4 cifras significativas.
    
    \item En notación científica, todos los dígitos del número principal son 
    significativos.  
    Ej.: $6.020 \times 10^{23}$ $\rightarrow$ 4 cifras significativas.
\end{enumerate}
\end{frame}

\begin{frame}
    \textbf{Ejemplo:} Si se mide una longitud como $12.34 \ \text{cm}$, las cuatro 
cifras indican precisión hasta la centésima de centímetro; el último dígito (4) 
es incierto, pero forma parte de la información significativa.
\end{frame}

\begin{frame}{Aproximaciones}
    A continuación se muestra la diferencia entre aproximar un número 
por \textbf{truncamiento} y por \textbf{redondeo} a diferentes cifras decimales.

\begin{center}
\begin{tabular}{cccc}
\toprule
\textbf{Valor original} & \textbf{Cifras decimales} & \textbf{Truncamiento} & \textbf{Redondeo} \\
\midrule
$3.14159$ & 4 & $3.1415$ & $3.1416$ \\
$3.14159$ & 3 & $3.141$  & $3.142$  \\
$3.14159$ & 2 & $3.14$   & $3.14$   \\
$3.14159$ & 1 & $3.1$    & $3.1$    \\
$3.14159$ & 0 & $3$      & $3$      \\
\bottomrule
\end{tabular}
\end{center}

\textbf{Explicación:}
\begin{itemize}
    \item En el \textbf{truncamiento} se cortan los dígitos después de la cifra deseada sin considerar su valor.
    \item En el \textbf{redondeo} se aumenta en una unidad la última cifra conservada si el siguiente dígito es 5 o mayor.
\end{itemize}
\end{frame}

\begin{frame}
\begin{center}
    \Huge ¿Preguntas?
\end{center}
\end{frame}

\begin{frame}
    \begin{figure}
        \centering
        \includegraphics[width=0.8\linewidth]{figures/meme-1.jpeg}
    \end{figure}
\end{frame}

\begin{frame}{Ejercicios propuestos}
\footnotesize
    \begin{table}[H]
    \centering
    \begin{tabular}{|c|c|M{1cm}|M{1cm}|M{1cm}|M{1cm}|}
    \hline
        Literal & Cantidad & Sin prefijo & Notación científica & Aprox. & Prefijo  \\\hline\hline
        A & $\num{7128433.076} \,\unit{km}$ & & &  & \\\hline
        B & $\num{0.00000003071} \,\unit{b}$ & & &  & \\\hline
        C & $\num{0.002716} \,\unit{mA}$ & & &  & \\\hline
        D & $\num{24182.33708} \,\unit{\mu^\circ C}$ & & &  & \\\hline
        E & $\num{88807166254} \,\unit{hb}$ & & &  & \\\hline
        F & $\num{0.0000300008} \,\unit{m}$ & & &  & \\\hline
        G & $\num{3821714321.66} \,\unit{pJ}$ & & &  & \\\hline
        H & $\num{6082417.9127} \,\unit{cl}$ & & &  & \\\hline
        I & $\num{0.0000000000001} \,\unit{Tg}$ & & &  & \\\hline
        J & $\num{421809718.006} \,\unit{cg}$ & & &  & \\\hline
        K & $\num{0.0000070128002} \,\unit{s}$ & & &  & \\\hline
        L & $\num{1111111100.20001} \,\unit{K}$ & & &  & \\\hline
    \end{tabular}
\end{table}
\end{frame}

\begin{frame}
\footnotesize
    \begin{table}[H]
    \centering
    \begin{tabular}{|c|c|M{1cm}|M{1cm}|M{1cm}|M{1cm}|}
    \hline
        Literal & Cantidad & Sin prefijo & Notación científica & Aprox. & Prefijo  \\\hline\hline
        M & $\num{0.000000000000000007} \,\unit{Pb}$ & & &  & \\\hline
        N & $\num{28421571.00382} \,\unit{\mu K}$ & & &  & \\\hline
        O & $\num{222341567.886} \,\unit{ms}$ & & &  & \\\hline
        P & $\num{718718210047216687} \,\unit{fs}$ & &  &  & \\\hline
        Q & $\num{800000000000000} \,\unit{g}$ & & &  & \\\hline
        R & $\num{330182.43} \,\unit{km}$ & & &  & \\\hline
        S & $\num{1050000233} \,\unit{b}$ & & &  & \\\hline
        T & $\num{0.0000000001} \,\unit{cm}$ & & &  & \\\hline
        U & $\num{323998417992.0006} \,\unit{m}$ & & &  & \\\hline
        V & $\num{8240017.83} \,\unit{cb}$ & & &  & \\\hline
        W & $\num{11717111111.17} \,\unit{\mu C}$ & & &  & \\\hline
        X & $\num{9999999999999} \,\unit{m}$ & & &  & \\\hline
        Y & $\num{2130000000} \,\unit{ml}$ & & &  & \\\hline
        Z & $\num{0.00000000082} \,\unit{GA}$ & & &  & \\\hline
    \end{tabular}
\end{table}
\end{frame}

\begin{frame}
\begin{center}
    {\Huge \textbf{VECTORES}}

    \vspace{1em}
    
    (¡ojo, que está en mayúscula!)
\end{center}
    
\end{frame}

\begin{frame}
    
    \begin{center}
    Sin rodeos:
    
    \vspace{2em}
    
        \LARGE \textbf{UN VECTOR ES UN NÚMERO DE VARIAS DIMENSIONES.}
    \end{center}
    
\end{frame}

\begin{frame}

\begin{center}
    \Huge FIN.
\end{center}

\end{frame}

\begin{frame}{¿Por qué vectores?}
    Algunas magnitudes físicas, como tiempo, temperatura, masa o densidad, se describen solo con un número y una unidad (cantidades \textit{escalares}). Sin embargo, otras, como el desplazamiento, la velocidad o la fuerza, requieren también una dirección para estar completamente definidas. Estas magnitudes con módulo y dirección se llaman \textit{vectoriales}.
\end{frame}

\begin{frame}{Tipos de magnitudes físicas}
    \begin{center}
\begin{tabular}{ll}
\toprule
\textbf{Escalares} & \textbf{Vectoriales} \\
\midrule
Tiempo          & Desplazamiento \\
Temperatura     & Velocidad \\
Masa            & Aceleración \\
Densidad        & Fuerza \\
Energía         & Momento lineal \\
Presión         & Campo eléctrico \\
Trabajo         & Campo magnético \\
Potencia        & Impulso \\
$$\vdots$$        & $$\vdots$$ \\
\bottomrule
\end{tabular}
\end{center}
\end{frame}

\begin{frame}{Ejercicios}
    Descomponga rectangularmente los siguientes vectores 
    
    \begin{multicols}{2}
        
        \begin{figure}[H]
            \includegraphics[width=0.4\textwidth]{figures/composicion-y-descomposicion-1.jpg}
        \end{figure}
        
        
        \begin{figure}[H]
            \includegraphics[width=0.4\textwidth]{figures/suma2.jpg}
        \end{figure}
        
        $$Q = 100 \text{ N}$$
        
    \end{multicols}
\end{frame}

\begin{frame}
\begin{center}
    \Large Vayamos al tablero para ver cómo funciona.
    \begin{figure}
    \centering
    \includegraphics[width=0.6\linewidth]{figures/imagen1.png}
\end{figure}
\end{center}
\end{frame}

\begin{frame}
    Se dice que un sistema est\'a en equilibrio traslacional si $\vec{F}_N = \vec{0}$. Discrimine el estado de equilibrio de los siguientes sistemas en los cuales actúan las fuerzas dadas.
    
    \begin{itemize}
        \item[a)] $\vec{F}_1 = (1000 \text{ N} , 37^\circ)$ y $\vec{F}_2 = (1000 \text{ N} , 217^\circ)$.
        \item[b)] $\vec{F}_1 = (1000 \text{ N} , 270^\circ)$ y $\vec{F}_2 = (800 \text{ N} , 90^\circ)$.
        \item[c)] $\vec{F}_1 = (440 \text{ N} , 225^\circ)$ y $\vec{F}_2 = (440 \text{ N} , 315^\circ)$.
        \item[d)] $\vec{F}_1 = (800 \text{ N} , 0^\circ)$ y $\vec{F}_2 = (800 \text{ N} , 180^\circ)$.
    \end{itemize}
\end{frame}

    \begin{frame}
    \begin{multicols}{2}
    \begin{figure}[H]
            \includegraphics[width=0.5\textwidth]{figures/Ejercicios-de-suma-de-vectores-1.png}
        \end{figure}
        
        \begin{figure}[H]
        \includegraphics[width=0.5\textwidth]{figures/vectores-30.png}
        \end{figure}
        
        $A = 40 \text{ N}$, $B = 75 \text{ N}$, $C = 50 \text{ N}$ y $D = 45 \text{ N}$.
\end{multicols}
        
    \end{frame}

\begin{frame}
\begin{center}
    \Huge \textbf{Capítulo 2}
    
    \LARGE Cinemática

    \textit{La descripción matemática del movimiento}
\end{center}
    
\end{frame}

\begin{frame}
    \begin{center}
        {\LARGE ¿Qué es la cinemática?}

        \vspace{2em}

        \textbf{En la siguiente diapositiva lo explico contundentemente.}
    \end{center}

    
\end{frame}

\begin{frame}
    \begin{figure}
    \centering
    \includegraphics[width=0.8\linewidth]{figures/meme2.jpeg}
\end{figure}
\end{frame}

\begin{frame}{Cinemática}
    En otras palabras...

    \vspace{1em}
    
    \begin{center}
        \textit{Estudia el movimiento de los cuerpos sin considerar las causas que lo producen.}
    \end{center}
    

    \vspace{1em}

    \textbf{Qué estudia:} el movimiento.

\textbf{Qué ignora:} las causas del movimiento (eso lo hace la dinámica).

\textbf{Magnitudes clave:} posición, desplazamiento, velocidad, aceleración, tiempo.
    
\end{frame}

\begin{frame}{Variables cinemáticas}
Las magnitudes físicas imperativas en la cinemática son las siguientes.

    \begin{itemize}
        \item \textbf{Tiempo:} magnitud fundamental que permite ordenar la secuencia de los sucesos (pasado, presente, futuro), medir la duración y la separación entre acontecimientos, y determinar su simultaneidad. Típicamente se denota con la letra $t$.
    \end{itemize}
\end{frame}

\begin{frame}
    \begin{itemize}
        \item \textbf{Posición:} Ubicación de un objeto en el espacio, relativo a un punto de referencia. Se mide en unidades de longitud. Típicamente se denota con la letra $x$ y, en general, es una función del tiempo $x=x(t)$.
        \item \textbf{Desplazamiento:} Cambio en la posición de un objeto: $$\Delta x = x(t_\text{final})-x(t_\text{inicial}).$$ Se mide en unidades de longitud.
        \item \textbf{Distancia:} Suma de los valores absolutos de los desplazamientos de un objeto. Típicamente se denota con la letra $d$. Está dada por $$d = \sum_{n=1}^N\lvert\Delta x\lvert_n.$$
    \end{itemize}
\end{frame}

\begin{frame}{Concepto de velocidad}
    Considerar una partícula que recorre espacios iguales en tiempos iguales. Es decir, su posición respecto de un punto de referencia se incrementa de forma directamente proporcional al tiempo.

    \begin{figure}
        \centering
        \includegraphics[width=0.4\linewidth]{figures/XvsT1.jpg}
    \end{figure}

    En esta situación, la \textbf{velocidad} se define como la variación de la posición respecto de la variación en el tiempo dentro de los mismos intervalos, es decir, $$v=\frac{x_f-x_0}{t_f-t_0}=\frac{x_j-x_i}{t_j-t_i}=\frac{\Delta x}{\Delta t}.$$
    
\end{frame}

\begin{frame}{Velocidad instantanea}

    \begin{figure}
        \centering
        \includegraphics[width=0.5\linewidth]{figures/XvsT2.jpg}
    \end{figure}
    
    En general, para un objeto que se mueva en el espacio, la velocidad instantanea se define como la variación infinitesimal de la posición respecto de un incremento diferencial en el tiempo. Esto es,
    \begin{equation}
        v=\frac{dx}{dt}.
    \end{equation}
\end{frame}

\begin{frame}{Gráficas de $v$ vs $t$}

Una gráfica de velocidad contra tiempo muestra la evolución temporal de la velocidad instantanea de un sistema en el tiempo. El área bajo la curva de la gŕafica de velocidad contra tiempo en un intervalo de tiempo dado es el desplazamiento del cuerpo en ese intervalo.

\begin{figure}
    \centering
    \includegraphics[width=0.4\linewidth]{figures/VvsT1.jpg}
\end{figure}

Por lo tanto, \begin{equation}
    x(t) = x_0 + \int_{t_0}^tv(t')\,dt'.
\end{equation}
    
\end{frame}

\begin{frame}{Posición de un sistema con velocidad constante}

Si un sistema se mueve con velocidad constante, esto es, $dv/dt=0$, su posición en función del tiempo está dada por la ecuación
    \begin{equation}
        x(t)=x_0+v(t-t_0).
    \end{equation}
    A este tipo de movimiento se le llama \textit{movimiento rectilineo uniforme}.
\end{frame}

\begin{frame}{Ejercicio 1}
    Considere un m\'ovil que se mueve en l\'inea recta sin aceleraci\'on. Se registra su posici\'on en diferentes instantes de tiempo tal que as\'i:
    \begin{itemize}
        \item Parte de $x = 2$ m y avanza 8 m hacia adelante en 4 s.
        \item Luego, retrocede 6 m en 3 s.
        \item Permanece en reposo en esa posici\'on durante 5 s.
        \item Finalmente, se mueve 10 m hacia adelante en 5 s.
    \end{itemize}
    
    Basado en la anterior descripci\'on,
    
    \begin{itemize}
        \item[a)] Grafique el movimiento descrito en una gr\'afica de posici\'on contra tiempo.
        \item[b)] Determine el desplazamiento en cada intervalo. 
        \item[c)] Calcule el desplazamiento total.
        \item[d)] Calcule la distancia total recorrida.
        \item[e)] Calcule la velocidad del móvil en cada tramo.
        \item[f)] Determine la ecuaci\'on de movimiento para cada tramo.
    \end{itemize}

\end{frame}

\begin{frame}{Ejercicio 2}
    Un m\'ovil inicia su movimiento hacia el este con una velocidad constante de $15 \,\text{m/s}$. 3.75 segundos despu\'es, un segundo m\'ovil parte hacia el oeste con velocidad constante de $25 \,\text{m/s}$. Si inicialmente los móviles distaban $20 \,\text{km}$, determine
    
    \begin{itemize}
        \item[a)] Cu\'anto tiempo les tomar\'a encontrarse.
        \item[b)] Cu\'al es la distancia recorrida por cada uno hasta el punto de encuentro.
        \item[c)] Cuánto tiempo le tomará al segundo alcanzar al primero si el primero inicialmente partiera hacia el oeste.
        \item[d)] Cuál es la distancia recorrida por cada uno en la situación del literal (c).
    \end{itemize}
\end{frame}

\begin{frame}{Concepto de aceleración}
    Considerar una partícula que al desplazarse, cambia su velocidad en incrementos iguales en tiempos iguales, es decir, el cambio de velocidad es directamente proporcional al cambio en el tiempo.

    \begin{figure}
        \centering
        \includegraphics[width=0.4\linewidth]{figures/VvsT2.jpg}
    \end{figure}

    En esta situación, la \textbf{aceleración} se define como la variación de la velocidad respecto de la variación en el timpo dentro de los mismos intervalos, es decir, $$a=\frac{v_f-v_0}{t_f-t_0}=\frac{v_j-v_i}{t_j-t_i}=\frac{\Delta v}{\Delta t}.$$
    
\end{frame}

\begin{frame}{Aceleración instantanea}


    \begin{figure}
        \centering
        \includegraphics[width=0.5\linewidth]{figures/VvsT3.jpg}
    \end{figure}
    
    En general, para un objeto que se mueva en el espacio, la aceleración instantanea se define como la variación infinitesimal de la velocidad respecto de un incremento diferencial en el tiempo. Esto es,
    \begin{equation}
        a=\frac{dv}{dt}.
    \end{equation}
\end{frame}

\begin{frame}{Gráficas de $a$ vs $t$}

Una gráfica de aceleración contra tiempo muestra la evolución temporal de la aceleración instantanea de un sistema en el tiempo. El área bajo la curva de la gŕafica de aceleración contra tiempo en un intervalo de tiempo dado es el cambio en la velocidad del cuerpo en ese intervalo.

\begin{figure}
    \centering
    \includegraphics[width=0.4\linewidth]{figures/AvsT1.jpg}
\end{figure}

Por lo tanto, \begin{equation}
    v(t) = v_0 + \int_{t_0}^ta(t')\,dt'.
\end{equation}

\end{frame}

\begin{frame}{Posición de un sistema con aceleración constante}

Si un sistema se mueve con aceleración constante, esto es, $da/dt=0$, su posición en función del tiempo está dada por la ecuación
    \begin{equation}
        x(t)=x_0+v_0(t-t_0)+\frac{1}{2}a(t-t_0)^2.
    \end{equation}
    A este tipo de movimiento se le llama \textit{movimiento rectilineo uniformemente acelerado}.

    
    Nótese que si $a=0$, la ecuación de movimiento se transforma en la misma ecuación del movimiento rectilineo uniforme.
\end{frame}

\begin{frame}{Relación entre posición y velocidad de un sistema con aceleración constante}
    Si un sistema se mueve con aceleración constante, esto es, $da/dt=0$, su posición en función de la velocidad está dada por \begin{equation}
        [v(x)]^2=v_0^2+2a(x-x_0).
    \end{equation} 
\end{frame}

\begin{frame}{Ejercicio 1}
    Un objeto A es arrojado verticalmente hacia arriba con una velocidad de $6$ ft/s. Justo cuando llega al punto más alto, otro objeto B es arrojado verticalmente hacia arriba con una velocidad de $3.2$ ft/s.
	    
	    Determine
	    
	    \begin{itemize}
	    \item[a)] El tiempo para el cual ambos cuerpos se encuentran en su recorrido.
	    \item[b)] La posición en la que se encuentran.
	    \item[c)] Qué sucederá primero: ¿El cuerpo A llega al suelo o el cuerpo B llega el punto más alto?
	    \end{itemize}
	    
\end{frame}

\begin{frame}{Idealización}
    En física, una idealización es una simplificación intencionada de un fenómeno o sistema real para hacerlo más manejable o fácil de analizar, asumiendo que ciertos detalles poco relevantes son irreales o pueden ignorarse. Por ejemplo: \begin{itemize}
        \item Una superficie muy lisa carece de rozamiento.
        \item Un objeto que cae en las profundidades del mar no se ve afectado por las corrientes marítimas
        \item Una superficie plana es perfectamente plana.
        \item Un agujero negro es perfectamente esférico (radio de Schwarzschild).
    \end{itemize}

    $$\vdots$$
\end{frame}

\begin{frame}
    \begin{figure}
        \centering
        \includegraphics[width=\linewidth]{figures/vaca-esferica.jpg}
    \end{figure}
\end{frame}

\begin{frame}
    \begin{figure}
        \centering
        \includegraphics[width=0.8\linewidth]{figures/spherical-chiken.png}
    \end{figure}
\end{frame}

\begin{frame}{Idealizaciones (hasta nuevo aviso)}

A continuación, se detallan una serie de idealizaciones que usaremos (hasta nuevo aviso) en el curso.

\begin{enumerate}
    \item Los objetos que se mueven no tienen dimensión.
    \item El aire no existe.
    \item Las superficies lisas son perfectamente lisas.
    \item La aceleración gravitacional es la misma en todos los lugares del planeta tierra.
\end{enumerate}

    
\end{frame}

\begin{frame}{Caida libre}
    Diálogos sobre dos nuevas ciencias (1638) es la última obra de Galileo Galilei, donde resume décadas de investigaciones. En ella, presenta dos pilares clave: la resistencia de los materiales y el movimiento de los cuerpos.

Su impacto en la física fue enorme, ya que:

\begin{itemize}
    \item Introdujo una descripción matemática del movimiento uniformemente acelerado.
    \item Sentó las bases de la cinemática moderna.
    \item Fue precursor del método científico experimental.

    \item Influenció directamente a Newton en la formulación de sus leyes.

    \item La obra marcó el tránsito de la física aristotélica a la física moderna.
\end{itemize}

\end{frame}

\begin{frame}

"$[\dots]$ porque así, como la uniformidad 
del movimiento se define y se concibe por medio de la 
uniformidad de los tiempos y de los espacios (pues al movimiento 
le llamamos uniforme, cuando espacios iguales son recorridos 
en tiempos iguales), así también, por medio de la 
igualdad, de los intervalos del tiempo, podemos concebir los 
incrementos de la velocidad simplemente agregados; entendiendo 
que ese movimiento es acelerado uniformemente y del 
mismo modo continuamente, siempre que en cualesquiera 
tiempos iguales se le vayan sobreañadiendo aditamentos iguales 
de velocidad. De modo que si, tomado un número cualquiera 
de intervalos iguales de tiempo, a contar desde el primer 
instante en que el móvil abandona el reposo y comienza el descenso, 
la velocidad, adquirida durante el primero más el segundo 
intervalo de tiempo, es doble de aquella que el móvil adquirió 
durante el primer intervalo solo; la velocidad que adquiere 
durante tres intervalos de tiempo, es triple; y la que adquiere 
en cuatro, cuádruple de la velocidad del primer tiempo. 
    
\end{frame}

\begin{frame}
    De 
modo que (para más clara comprensión), si el móvil continua 
a su movimiento uniformemente con la velocidad adquirida 
en el primer intervalo de tiempo, este movimiento sería dos 
veces más tardo que aquel que hubiera alcanzado con la velocidad 
adquirida en dos intervalos de tiempo. Y así, no parece 
repugnar a la recta razón el admitir que el incremento de la velocidad 
se efectúa según la extensión del tiempo; de donde, la
definición del movimiento que vamos a tratar, puede ser la siguiente: 

\bigskip

\begin{itemize}
    \item[] \textit{Llamo movimiento igualmente o uniformemente acelerado 
aquel que, a partir del reposo, va adquiriendo incrementos 
iguales de velocidad durante intervalos iguales de tiempo.}"
\end{itemize}

\bigskip
\bigskip

\raggedleft{{\textit{Diálogos sobre dos nuevas ciencias}, \textbf{Galileo Galilei (1638)}}}

\end{frame}

\begin{frame}{Conclusión clave}
    \LARGE \textbf{Cuando un objeto cae en el seno de un campo gravitacional uniforme, describe un movimiento uniformemente acelerado.}
\end{frame}

\begin{frame}{Ecuaciones de movimiento}
    La ecuación de movimiento para un objeto que asciende o desciende en caida libre está dada por \begin{equation}
        y(t)=y_0+v_0(t-t_0)-\frac{1}{2}g(t-t_0)^2,
    \end{equation} donde $g$ es la aceleración gravitacional y la referencia $y=0$ está en el punto más bajo del espacio (normalmente, el suelo).
    
    \vspace{1em}
    
    En el planeta Tierra, $$g=\num{9.81}\,\unit{m}/\unit{s}^2.$$
\end{frame}

\begin{frame}{Ejercicio 1}
    Un tubo que gotea, libera gotas de agua cada 3 s. Determine la distancia que separará a dos gotas consecutivas, pasados 7.2 s desde que cae la primera gota.
\end{frame}

\begin{frame}{Ejercicio 2}
    Un hombre se mueve en  motocicleta con rapidez 
constante de 54 $\unit{km}/\unit{h}$ hacia un 
edificio de 40 m de altura. Una 
persona en la azotea del edificio 
lanza verticalmente una pelota con 
la intención de que caiga en una 
canasta situada a 1 m del suelo en 
la motocicleta. La persona en la 
azotea lanza la pelota desde una 
altura de 42 m y la motocicleta está 
a 30 m del edificio. Determine la velocidad con la que debe lanzarse la pelota para que esta caiga en 
la canasta cuando la motocicleta está en la base del edificio.

\begin{figure}
    \centering
    \includegraphics[width=0.7\linewidth]{figures/moto-edificio.png}
\end{figure}

\end{frame}

\begin{frame}{Movimiento parabólico}
\begin{figure}
    \centering
    \includegraphics[width=0.7\linewidth]{figures/Parabolico.jpg}
\end{figure}
    El movimiento parabólico es un tipo de movimiento bidimensional en el que la ubicación del objeto que describe tal movimiento está dada por un vector de posición dependiente del tiempo, $$\vec{r}(t)=x(t)\hat{\imath}+y(t)\hat{\jmath}.$$
\end{frame}

    \begin{frame}{Posición en el movimiento parabólico}
    
    La evolución temporal del vector de posición está dada tal que así: \begin{itemize}
        \item La componente horizontal evoluciona de acuerdo con un MRU, \begin{equation}
            x(t)=x_0+v_x(t-t_0).
        \end{equation}
        \item La componente vertical evoluciona temporalmente de acuerdo con un MRUA influenciado por la aceleración gravitacional, \begin{equation}
            y(t)=y_0+v_{0y}(t-t_0)-\frac{1}{2}g(t-t_0)^2.
        \end{equation}
    \end{itemize}
    El vector de velocidad evoluciona como

    \begin{align}
        \nonumber v(t)&=\frac{d\vec{r}}{dt}=\frac{dx}{dt}\hat{\imath}+\frac{dy}{dt}\hat{\imath}=v_x\hat{\imath}+v_y\hat{\jmath}\\
        &=v_x\hat{\imath}+(v_{0y}-g(t-t_0))\hat{\jmath}.
    \end{align}
\end{frame}

\begin{frame}{Ejercicio 1}
    Un proyectil arrojado con un ángulo de $\pi/4\,\unit{rad}$ sobre la horizontal, a una velocidad de 9 $\unit{m}/\unit{s}$. Determine:
	
	\begin{itemize}
	    \item[a)] La altura máxima que alcanza.
	    \item[b)] El tiempo de vuelo.
	    \item[c)] El desplazamiento horizontal máximo.
	\end{itemize}
\end{frame}

\begin{frame}{Ejercicio 2}
 Un pasajero en un 
tren que se mueve a 80 
$\unit{km}/\unit{h}$ lanza verticalmente 
hacia arriba una pelota 
desde una ventana 
situada a 1.5 m de altura 
respecto del suelo con 
una velocidad de 6 $\unit{m}/\unit{s}$. 
La pelota debe caer en 
una caja de 1.0 m de altura y 50 cm de ancho, situada a 25 m horizontales del punto de 
lanzamiento. ¿Cae la pelota dentro de la caja?

\begin{figure}
    \centering
    \includegraphics[width=0.8\linewidth]{figures/tren-caja.png}
\end{figure}
   
\end{frame}